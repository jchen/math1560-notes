%!TEX root = ../notes.tex
\section{April 21, 2022}
\recall last time we\dots
\begin{itemize}
    \item Discussed fractional ideals (in $\riO_K$).
    \item Introduced inverses to nonzero fractional ideals in $\riO_K$/Dedekind domains.
    \item Proved $\riO_K$ was a Dedekind domain
    \item Used the properties of $\riO_K$ to show that every nonzero ideal of $\riO_K$ factors uniquely as a product of prime ideals (and the same argument works for Dedekind domains).
\end{itemize}

\subsection{Ramification Theory}
\begin{figure}[ht!]
    \begin{center}
        \includegraphics[width=0.3\textwidth]{images/kirby.png}
    \end{center}
    \caption{Kirby!}
    \begin{ques*}
        Kirby, what primes \ul{ramify} in $\QQ(i)$?
    \end{ques*}
\end{figure}

A major topic/theme in classical algebraic number theory is the factorization of ideals in $\riO_K$ generated by primes in $\ZZ$.
\begin{definition}
    Let $p\in\ZZ_+$ be a prime, and let $K$ be a number field.

    \begin{enumerate}[1)]
        \item We say that $p$ \ul{ramified} in $K$ (or $\riO_K$) if, for
              \[(p) := p\cdot\riO_K = \frk{p}_1^{e_1}\frk{p}_2^{e_2}\cdots \frk{p}_r^{e_r},\]
              (for $\frk{p}_i$'s pairwise distinct) we have that some $e_i\geq 2$.
        \item We say that $p$ is \ul{totally ramified} if
              \[(p) = \frk{p}_1^n\]
              where $n = [K : \QQ]$ and $\frk{p_1}$ is prime.
        \item We say that $p$ is \ul{intert} if $(p)$ is a prime ideal of $\riO_K$.
        \item We say that $p$ is \ul{totally split} if
              \[(p) = \frk{p}_1 \frk{p}_2\cdots \frk{p}_r\]
              (for $\frk{p}_i$'s pairwise distinct), $n = [K : \QQ]$.
    \end{enumerate}
\end{definition}
\begin{remark}
    These categories are not \emph{all} the classifications of primes!
\end{remark}
\begin{example}
    $(2) = (1 + i)^2$ in $\ZZ[i] = \riO_K$ for $K = \QQ(i)$, so $2$ \emph{ramifies} in $K$ (and in fact is \emph{totally ramified}).
\end{example}
\begin{example}
    $(3)$ in $\ZZ[i]$? It turns out that $(3)$ is a prime ideal. So $3$ is inert in $\QQ(i)$.
\end{example}
\begin{example}
    What about $(5)$ in $\ZZ[i]$?
    \[5 = (1 + 2i)(1 - 2i)\]
    and $1 + 2i$, $1 - 2i$ are irreducible, but are not associates. So $5$ is \emph{totally split} in $\QQ(i)$.
\end{example}
$2$ key structural questions are:
\begin{enumerate}[1)]
    \item Which primes of $\riO_K$ ramify?
    \item How do individual rational primes factor in $\riO_K$?
\end{enumerate}

\begin{theorem}
    $p$ is ramified in $K$ iff $p\mid \Disc(K)$.
\end{theorem}
\begin{example}
    We can now answer the question posed to Kirby! In $\QQ(i)$,
    \[\disc(\QQ(i)) = \left|\begin{array}{rr}
            1 & i \\ 1 & -i
        \end{array}\right| = (-2i)^2 = -4\]
    so $2$ is the only prime that ramifies in $\QQ(i)$.
\end{example}
We prove this in the monogenic case, i.e. when $\riO_K = \ZZ[\theta]$ for some $\theta\in\riO_K$.

The moral is: study the factorization of $f$ modulo $p$ where $f = \minpoly_\QQ(\theta)$.
\begin{proof}
    Suppose $K$ is monogenic, with $\riO_K = \ZZ[\theta]$ and let $p\in\ZZ_+$ be prime. Let $f = \minpoly_\QQ(\theta)$. Since $\Disc(K) = \Delta[1, \theta, \theta^2, \dots, \theta^{n-1}] = \disc(f)$. We show that $p\mid \disc(f) \Leftrightarrow p$ ramifies in $K$.

    Let $p\cdot \riO_K = \frk{p}_1^{e_1}\frk{p}_2^{e_2}\cdots \frk{p}_r^{e_r}$ be the prime factorization of $p\cdot \riO_K$. Then
    \[\riO_K/(p) \cong \riO_K/\frk{p}_1^{e_1}\times \riO_K/\frk{p}_2^{e_2}\times \cdots\times \riO_K/\frk{p}_r^{e_r}\]
    by Sunzi's Theorem, as $\frk{p}_i + \frk{p}_j = \riO_K$, $\forall i\neq j$ (in HW7).

    \otoh, we have that
    \[\riO_K/(p) = \ZZ[\theta]/p\ZZ[\theta]\cong \ZZ[x]/(p, f(x))\cong (\ZZ/p\ZZ)/(\overline{f}(x))\]
    We first have that $\ZZ[\theta]\cong \ZZ[x]/f(x)$ (since $f$ is the minimum polynomial of $\theta$), then we quotient again by $p$\footnote{Since we have $(R/(a))/(\overline{b}) \cong R/(a, b)$. For example, $(\ZZ/5\ZZ)/(\overline{2})\cong\ZZ/(2, 5)\cong \ZZ/\ZZ$.}.

    If $\overline{f}(x) = \overline{\pi}_1(x)^{f_1}\overline{\pi}_2(x)^{f_2}\cdots \overline{\pi}_s(x)^{f_s}$ is the factorization of $\overline{f}$ into produce of irreducibles, then
    \[(\ZZ/p\ZZ)[x]/(\overline{f})\cong \FF_p[x]/\overline{\pi}_1(x)^{f_1}\times \cdots \times \FF_p[x]/\overline{\pi}_s(x)^{f_s}\]
    (also by Sunzi's theorem.)

    Our goal is to show that these line up\dots $s = r$ and exponents line up.

    We isolate $\riO_K / \frk{p}_1^{e_1}$ and we have chain
    \[\frk{p}_1^{e_1}\subsetneq \frk{p}_1^{e_1}\subsetneq \cdots \subsetneq \frk{p}_1 \subsetneq \riO_K\]
    so $\frk{p}_1 / \frk{p}_1^{e_1}$ is a maximal ideal of $\riO_K/\frk{p}_1^{e_1}$.

    To find maximal ideals in $\riO_K/(p)$, we take the maximal ideal in the first product ($\frk{p}_1 / \frk{p}_1^{e_1}$) and product with the rest of the product. So there are exactly $r$ maximal ideals, and on the other side we have exactly $s$ maximal ideals. Thus $r = s$.

    We do a similar thingy done in 1530 when we showed every finite Abelian group is the product of some cyclic groups:
    \begin{align*}
        G & \cong \ZZ/p_1^{e_1}\ZZ \times \cdots \times \ZZ/p_r^{e_r} \\
          & \cong \ZZ/q_1^{f_1}\ZZ \times \cdots \times \ZZ/q_s^{f_s}
    \end{align*}
    where we reordered. We start with some ideal $\mathfrak{I}_1$
    \[\mathfrak{I}_1 = I_1\times \riO_K/\frk{p}_2^{e_2}\times \cdots \times \riO_K/\frk{p}_r^{e_r}\]
    where $I_1$ is maximal in $\riO_K/\frk{p}_1^{e_1}$, and we start forming chains to recover $e_1$, and we continue doing so by `carving' into other factors. (``nilpotence argument'' shows that after appropriate reordering, $e_i = f_i$, $\forall i$.)

    We want to characterize when $e_i\geq 2$ (so we have ramification). This is when some $f_i\geq 2$, which is to say whether $\overline{f}$ mod $p$ has a repeated root (is not separable). But this is equivalent to saying
    \[\disc(f)\equiv 0\pmod{p}\]
    since taking discriminants commutes with reduction modulo $p$. So $p\mid \disc(f) = \Disc(K)$.
\end{proof}