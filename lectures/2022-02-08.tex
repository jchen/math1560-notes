%!TEX root = ../notes.tex
\section{February 8, 2022}

\subsection{Dirichlet Convolutions}

\begin{definition}[Dirichlet Convolution]
    Let $f, g$ be arithmetic functions. Then the \ul{Dirichlet convolution/product} of $g$ and $g$ it
    \begin{align*}
        (f * g)(n) : & = \sum_{d_1d_2=n}f(d_1)g(d_2) \\
                     & = \sum_{d\mid n}f(d) g(n/d)
    \end{align*}
\end{definition}

We do check that this has properties that we want it to have, like associativity:
\begin{align*}
    ((f * g) * h)(n) & = (f * (g * h))(n)                      \\
                     & = \sum_{d_1d_2d_3=n} f(d_1)g(d_2)h(d_3)
\end{align*}
It is also clearly commutative.

We also have that this product has a multiplicative identity.

\begin{definition}
    Let $I : \ZZ_+\to \{0, 1\}$ be given by
    \[I(n) = \begin{cases}
            1 & \text{if } n = 1 \\
            0 & \text{otherwise}
        \end{cases}\]
    Then $I$ is an identity for $*$, in the sense that $f * I = I * f = f$.
\end{definition}

\begin{lemma}
    If $f$ is an arithmetic function such that $f(1)\neq 0$, then there exists an arithmetic function $g$ such that $f * g = I$.

    It is given recursively by
    \begin{align*}
        g(1) & = \frac{1}{f(1)}                                         \\
        g(n) & = -\frac{1}{f(1)}\cdot \sum_{d\mid n, d < n} g(d) f(n/d)
    \end{align*}
\end{lemma}
\begin{proof}
    We want to show that given $g$ and $g$ defined as above, we have that $f * g = I$.

    $n=1$: \[g(1)\cdot f(1) = \frac{1}{f(1)}f(1) = 1\]

    $n > 1$: \begin{align*}
        \sum_{d\mid n}g(d) f(n/d) & = g(n)\cdot f(1) + \sum_{d\mid n, n<n}g(d) f(n/d)                   \\
                                  & = -\frac{1}{f(1)}\cdot \sum_{d\mid n, d < n} g(d) f(n/d) \cdot f(1)
        + \sum_{d\mid n, n<n}g(d) f(n/d)                                                                \\
                                  & = 0
    \end{align*}
    So $g$ is indeed an inverse of $f$ since they produce the identity function $I$.
\end{proof}

\subsection{M\"obius Inversion}
The motivation of this is: given a summatory function of multiplicative functions, can we recover the multiplicative function?
\begin{definition}[M\"obius $\mu$ Function]
    We define $\mu : \ZZ_+ \to \{-1, 0, 1\}$ given by
    \[\mu(n) = \begin{cases}
            (-1)^k & \text{if $n = p_1p_2\dots p_k$ if $p_i$ are pairwise distinct primes} \\
            0      & \text{otherwise}
        \end{cases}\]
    \emph{(We note that $\mu(1) = 1$.)}
\end{definition}

\begin{lemma}
    $\mu$ is a multiplicative function.
\end{lemma}
\begin{proof}
    Let $m, n\in\ZZ_+$ such that $(m, n) = 1$. We write
    \begin{align*}
        m & = p_1^{e_1}p_2^{e_2}\cdots p_k^{e_k} \\
        n & = q_1^{f_1}q_2^{f_2}\cdots q_l^{f_l} \\
    \end{align*}
    \begin{description}
        \item[Case 1]
            Some exponent $e_i$ or $f_i \geq 2$. Then we have that \[\mu(mn) = \mu(m)\mu(n) = 0\]
        \item[Case 2]
            We have that \begin{align*}
                m & =p_1p_2\cdots p_k \\
                n & =q_1q_2\cdots q_l
            \end{align*}
            where $p_i$ and $q_i$ are all pairwise distinct. Then $\mu(m) = (-1)^k$ and $\mu(n) = (-1)^l$, so $\mu(m) = \mu(n) = (-1)^{k+l}$.

            Since these are coprime $(m, n) = 1$, then we have that $\mu(mn) = (-1)^{k+l}$.
    \end{description}
    Which is as intended, giving that $\mu$ is a multiplicative function.
\end{proof}

\begin{lemma}
    We have the property:
    \begin{align*}
        \sum_{d\mid n}\mu(d) = 0 \qquad \forall n\geq 2.
    \end{align*}
    Which tells us that the summatory function of $\mu$ is $I$.
\end{lemma}
\begin{proof}
    We define
    \begin{align*}
        f(n) := \sum_{d\mid n}\mu(d) \qquad \text{is multiplicative}
    \end{align*}
    We check this on prime powers, for prime $p$ and $e\geq 1$:
    \begin{align*}
        f(p^e) & = \mu(1) + \mu(p) + \mu(p^2) + \cdots + \mu(p^e) \\
               & = 1 - 1 + 0 + \cdots + 0 = 0
    \end{align*}
    so we're done since $f$ is multiplicative and is $0$ for all power of primes.
\end{proof}

\begin{lemma}
    Let $i: \ZZ_+\to \{1\}$ be the constant $1$ function.
    \[i * \mu = \mu * i = I\]
\end{lemma}
\begin{proof} In the case of $n=1$, we have $i(1)\mu(1) = 1$.

    For $n > 1$, we have $(i * \mu)(n) = \displaystyle \sum_{d\mid n}\mu(d) = 0$ from above.
\end{proof}

We see here that summatory functions can be seen as Dirichlet products: the summatory function $F$ of $f$ is $F = f * i$. What we said about summatory functions being multiplicative boils down to Dirichlet convolutions preserving multiplicativity.

\recall that summatory functions inherit multiplicativity. In fact, this holds for Dirichlet products as well. If $f, g$ are multiplicative, then so is $f * g$.

The proof is parallel to the proof for summatory functions, for \cref{lemma:summatory-preserves-multiplicativity}.

\begin{theorem}[M\"obius Inversion]
    Let
    \begin{align*}
        F(n) & = \sum_{d\mid n} f(d)                         \\
        \intertext{Then we have }
        f(n) & = \sum_{d\mid n}\mu(d)\cdot F(n/d) = \mu * F.
    \end{align*}
\end{theorem}
\begin{proof}
    $F = f * i$, then \[F * \mu = (f * i) * \mu = f * (i * \mu) = f * I = f.\]
    which was simpler than I expected\dots
\end{proof}

\begin{corollary}\label{cor:summatory-multiplicativity-original-multiplicative}
    If $F$ is the summatory function of $f$, and $F$ is multiplicative, then $f$ is also multiplicative, as $f = \mu * F$ and $\mu$ is multiplicative and convolutions with multiplicative functions are multiplicative.
\end{corollary}
\begin{corollary}
    \Cref{cor:summatory-multiplicativity-original-multiplicative} gives another proof that $\phi$ is multiplicative, as \[\sum_{d\mid n}\phi(d) = \phi * i = \mathrm{id}.\]
\end{corollary}

\subsection{Applications of M\"obius Inversion}
\subsubsection{Cyclotomic Polynomials}

\recall the $n^\mathrm{th}$ cyclotomic polynomial $\Phi_n(x)$ is the unique irreducible polynomial in $\ZZ[x]$ dividing $x^n - 1$ but no $x^k - 1$ for $k < n$.

Thus \[\Phi_n(x) = \prod_{\substack{1\leq k < n \\ (k, n) = 1}} \left(x - e^{2\pi i k / n}\right)\]
as the roots of this polynomial are exactly the primitive $n^\mathrm{th}$ roots of unity. We have that \[\prod_{d\mid n}\Phi_d(x) = x^n - 1.\]
By M\"obius inversion, if
\begin{align*}
    G(n) & = \prod_{d\mid n}g(d),
    \intertext{then we have that}
    g(n) & = \prod_{d\mid n}G(d)^{\mu(n/d)}
\end{align*}
In particular, taking $G(n) = x^n - 1$ (with particular $x\in\CC$) as an arithmetic function, we have \[\Phi_n(x) = \prod_{d\mid n}(x^d - 1)^{\mu(n/d)}\qquad (x\in \CC)\]
Applying this identity for enough $x\in \CC$ yields this as an identity of polynomials.

\subsubsection{Dynatomic Polynomials}
The roots of cyclatomic polynomials are roots of unity. Dynatomic polynomials have as roots the periodic points (of certain periods) of a polynomial.

\begin{definition}
    Let $K$ be a field, and let $f\in K[x]$ of degree $d\geq 2$. Let \[f^n = \underbrace{f\circ f\circ \cdots \circ f}_{n\text{ times}}\]
    then $P\in \overline{K}$\footnote{Algebraic numbers in field $K$, the field you get by adjoining all roots of polynomials in $K[x]$.} is said to be \ul{periodic} under $f$ if
    \[f^n(P) = P\qquad \text{for some $n\geq 1$}\]
\end{definition}
\begin{example}
    Let $f(x) = x^2 - 1$. $0$ is a period point under $f$:
    \[0\longmapsto -1\longmapsto 0\]
    and its \ul{period} is $2$.

    \begin{remark*}
        If $n$ is the smallest positive integer such that $f^n(p) = p$ ($p$ periodic), then we call $n$ the \ul{exact period} of $p$ under $f$.
    \end{remark*}
\end{example}
\begin{definition}
    The $n^\mathrm{th}$ dynatomic polynomial of $f$ is 
    \[\Phi_{f,n}(x) := \prod_{d\mid n}\left(f^d(x) - x\right)^{\mu(n/d)}\]
\end{definition}
We hope that $\Phi_{f, n}(x)$ has as its roots the points of exact period $n$\dots This hope is dashed\dots 
\begin{example}
    $f(x) = x^2 - \frac{3}{4}$. 
    \begin{align*}
        f^2(x) - x &= \left(x - \frac{3}{2}\right)\left(1 - \frac{1}{2}\right)^3 \\
        f(x) - x &= \left(x-\frac{3}{2}\right)\left(x + \frac{1}{2}\right)
    \end{align*}
    Thus \[\frac{f^2(x) - x}{f(x) - x} = \left(x + \frac{1}{2}\right)^2\]
    But $x=-\frac{1}{2}$ is \ul{fixed} under $f$. 
\end{example}