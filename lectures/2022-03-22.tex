%!TEX root = ../notes.tex
\section{March 22, 2022}
\subsection{Discriminants of bases, Vandermonde determinants}
We have some review from last time:

Let $K = \QQ(\theta)$ be a number field of degree $n$, let $\{\alpha_1, \dots, \alpha_n\}$ be a $\QQ$-basis of $K$, and let $\sigma_i : K\hookrightarrow \CC$, $1\leq i\leq n$ be the embeddings of $K$ into $\CC$.

The discriminant of $\{\alpha, \dots, \alpha_n\}$ is
\[\Delta[\alpha_i, \dots, \alpha_n] = \det\begin{pmatrix}
        \sigma_1(\alpha_1) & \sigma_1(\alpha_2) & \cdots & \sigma_1(\alpha_n) \\
        \sigma_2(\alpha_1) & \sigma_2(\alpha_2) & \cdots & \sigma_2(\alpha_n) \\
        \vdots             & \vdots             & \ddots & \vdots             \\
        \sigma_n(\alpha_1) & \sigma_n(\alpha_2) & \cdots & \sigma_n(\alpha_n) \\
    \end{pmatrix}^2\]

If $\{\beta_1, \dots, \beta_n\}$ is another basis, then for all $1\leq k\leq n$,
\[\beta_k = \sum_{i=1}^n c_{ik}\alpha_i, \quad c_{ik}\in \QQ,\]
where $\det(c_{ik})\neq 0$. Fact from homework is that
\[\Delta[\beta_i, \dots, \beta_n] = \det(c_{ik})^2\cdot \Delta[\alpha_i, \dots, \alpha_n]\]

\begin{theorem*}[2.7, p.42 \cite{stewart2015algebraic}]
    The discriminant of any $\QQ$-basis for $K$ is rational and nonzero.
\end{theorem*}
\begin{proof}
    It suffices to prove this for $\{1, \theta, \theta^2, \dots, \theta^{n-1}\}$.
\end{proof}

We have the following observation:
\begin{definition*}[Vandermonde Matrix]
    A (square) \ul{Vandermonde matrix} is a matrix of the form
    \[V = \begin{pmatrix}
            1      & t_1    & t_1^2  & \dots  & t_1^{n-1} \\
            1      & t_2    & t_2^2  & \dots  & t_2^{n-1} \\
            \vdots & \vdots & \vdots & \ddots & \vdots    \\
            1      & t_n    & t_n^2  & \dots  & t_n^{n-1} \\
        \end{pmatrix}\]
\end{definition*}

We then claimed (without proof) that
\begin{claim*}
    The determinant of $V$ is
    \[\prod_{1\leq i < j\leq n}(t_j - t_i).\]
\end{claim*}
\begin{proof}
    We know that $\det(V) = 0$ when $t_i = t_j$ for some $i\neq j$. So, $\det(V)$ (as a polynomial in $t_1, \dots, t_n$) is divisible by $t_i - t_j$ for $i < j$. We have that the total degree of $\det(V)$ as a polynomial in $t_1, \dots, t_n$ is
    \[\sum_{i=1}^{n-1} i = \frac{n(n-1)}{2}.\]
    On the other hand, the total degree of $D$ is also $\binom{n}{2} = \frac{n(n-1)}{2}$. Hence, $\det(V)$ is a scalar multiple of $D$ (since ).

    But $\det(V)$ and $D$ are both monic as polynomials (\emph{kinda}) in $\QQ(t_2, \dots, t_n)[t_1]$. Thus $\det(V) = D$. \emph{Or something like that}.
\end{proof}

Going back to the proof of \cref{thm:2.7}, we take $t_i = \theta_i := \sigma_i(\theta)$ to get
\begin{align*}
    \Delta[1, \theta, \dots, \theta^{n-1}] & = \prod_{i < j}(theta_i - \theta_j)^2  \\
                                           & = \mathsf{disc}(\minpoly_\QQ(\theta)).
\end{align*}
which is clearly rational and nonzero in $\QQ^+$.

\begin{example}
    Let
    \[K = \QQ(\sqrt{5})\]
    with the obvious basis $\{1, \sqrt{5}\}$. We have
    \begin{align*}
        \Delta[1, \sqrt{5}] & = \left|\begin{array}{cc}
                                          1 & \sqrt{5}  \\
                                          1 & -\sqrt{5}
                                      \end{array}\right|^2 \\
                            & = (-2\sqrt{5})^2             \\
                            & = 20
    \end{align*}
    and another basis is $\left\{1, \frac{1+\sqrt{5}}{2}\right\}$ and
    \begin{align*}
        \Delta\left[1, \frac{1+\sqrt{5}}{2}\right] & = \left|\begin{array}{cc}
                                                                 1 & \frac{1+\sqrt{5}}{2}  \\
                                                                 1 & -\frac{1+\sqrt{5}}{2}
                                                             \end{array}\right|^2 \\
                                                   & = (-2\frac{\sqrt{5}}{2})^2                      \\
                                                   & = (\sqrt{5})^2 = 5
    \end{align*}
\end{example}

\begin{example}
    Let
    \[K = \QQ(\sqrt[3]{2})\]
    A basis is $\{1, \sqrt[3]{2}, (\sqrt[3]{2})^2\} =: B$
    \begin{align*}
        \Delta(B) = \left|\begin{array}{ccc}
                              1 & \sqrt[3]{2}          & (\sqrt[3]{2})^2          \\
                              1 & \zeta_3\sqrt[3]{2}   & (\zeta_3\sqrt[3]{2})^2   \\
                              1 & \zeta_3^2\sqrt[3]{2} & (\zeta_3^2\sqrt[3]{2})^2
                          \end{array}\right|
    \end{align*}
    \emph{computation left as exercise\dots}
\end{example}

\subsection{Algebraic Integers}
\begin{definition}[Algebraic Integer]
    A complex number is an \ul{algebraic integer} if it is a root of a \emph{monic} polynomial with integer coefficients.
\end{definition}

We denote the set of algebraic integers by $\overline{\ZZ}$. By definitions, $\overline{\ZZ}\subseteq\overline{\QQ}$.

\begin{example}
    The following are some algebraic integers: 
    \begin{itemize}
        \item $\sqrt{2}\in\overline{\ZZ}$ since $\sqrt{2}$ is a root of $x^2 - 2$. 
        \item $\tau = \frac{1}{2}(1 + \sqrt{5})$, since $\tau^2 - \tau - 1 = 0$. 
    \end{itemize}
    Non-examples: 
    \begin{itemize}
        \item $\frac{22}{7}$ is not an algebraic integer. \emph{Why?} We look at the $7$-adic valuation of the monic polynomial when we plug $\frac{22}{7}$ in. What are some other ways to reason about this? 
    \end{itemize}
\end{example}
Key algebra fact: If $f(x)\in \ZZ[x]$ is a monic polynomial with $f(x) = g(x)\cdot h(x)$ where $g(x)$, $h(x)$ are monic polynomials in $\QQ[x]$, then $g(x), h(x)\in \ZZ[x]$. (Gauss's Lemma). 

Hence, if $\frac{22}{7}$ is an algebraic integer, if has some $f(x)\in\ZZ[x]$ for which it is a root. But it is also a root of $g(x) = x - \frac{22}{7}$ where $f(x) = g(x)\cdot h(x)$ forcing $g(x)$ to be in $\ZZ[x]$. So the fact that $\minpoly_\QQ\left(\frac{22}{7}\right) = x - \frac{22}{7}$ implies that $\frac{22}{7}\not\in\overline{\ZZ}$. 

\begin{definition*}[Algebraic Integer']
    An algebraic number $\theta$ is an algebraic integer iff $\minpoly_\QQ(\theta)\in\ZZ[x]$. 
\end{definition*}