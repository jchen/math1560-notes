%!TEX root = ../notes.tex
\setcounter{section}{-1}
\section{January 27, 2022}
\subsection{Course Logistics}
\begin{itemize}
    \item Mostly refer to syllabus for any information that you might need. 
    \item Midterm is planned for March 17. 
    \item Final exam schedule can be found on CAB. 
\end{itemize}

\subsection{Introduction to Number Theory}
Number theory can be split into two branches: analytic number theory and algebraic number theory. 

\emph{What is number theory?} Number theory is the study of integers and their analogues in algebraic number fields. 

Prime numbers are a key focus of number theory, and the study of different properties of primes constitutes different fields of number theory: 
\begin{enumerate}[i.]
    \item The study of their distributional properties, which is \ul{analytic number theory}. 
    \item As building blocks for algebraic numbers, which is \ul{algebraic number theory}. 
\end{enumerate}

\subsubsection{Examples of Analytic Number Theory}
Here are some examples of analytic number theory and their statements: 
\begin{itemize}
    \item Prime Number Theorem
    \item Twin Prime Conjecture
    \item Goldbach's conjecture
\end{itemize}
\begin{theorem}[Prime Number Theorem]
    Let $\pi(x)$ be the number of primes between $1$ and $x$, then
    \begin{equation*}
        \lim_{x\to\infty} \frac{\pi(x)}{x / \ln(x)} = 1.
    \end{equation*}
\end{theorem}
\begin{conjecture}[Twin Prime Conjecture]
    Twin primes are pairs of primes $p, q$ of the form $q=p+2$. Examples include $(3, 5), (11, 13), \dots$. The conjecture postulates that there are infinitely many twin primes. 
\end{conjecture}
\begin{conjecture}[Goldbach's conjecture]
    Any positive even integer greater than $2$ can be written as the sum of $2$ primes. 
\end{conjecture}

\subsubsection{Examples of Algebraic Number Theory}
Analyzing the factorization (rings of integers) of number fields is one topic of algebraic number theory. 
\begin{example}
    $2$ is prime (irreducible) in $\ZZ$. 

    Yet $2$ is not prime in $\ZZ[i]$ (the Gaussian integers). This is because 
    \[2 = \underbrace{(1+i)(1-i)}_{\text{associates}}\]
    we have that $(1+i) = i(1-i)$. We also note the property that the principal ideals $(2) = (1+i)^2$ are equal. 

    In this example, we say that $2$ ``ramifies'' in the ring of integers. 
\end{example}

Fermat's Last Theorem is another such example. 

\recall that a \emph{Pythagorean triple} is a triple of the form $a, b, c\in \ZZ_+$ such that 
\[a^2 + b^2 = c^2\]
Are there examples of such numbers with different exponents (say, $k^\mathrm{th}$ powers for $k\geq 3$)? 
\begin{theorem}[Fermat's Last]
    There are no positive integers $a, b, c\in \ZZ_+$ satisfying 
    \[a^k + b^k = c^k\]
    for $k\geq 3$. 
\end{theorem}
The answer is no! (Proved by Andrew Wiles)

\begin{conjecture}[$abc$ Conjecture, informally]
    We say \emph{powerful numbers} are positive integers whose prime factorization contains relatively few distinct primes (appropriately weighted) with an exponent of 1. 
    \begin{example*}
        $2^{10}3^7$ is powerful, $2^{10}3^{7}5$ is powerful, $1$ is powerful. 
    \end{example*}

    If $a, b$ are \emph{very powerful} coprime numbers, then $a+b$ is predicted to be \emph{not powerful}. 
\end{conjecture}
\begin{example}
    Consider $2^{10}$ and $3^{15}$. We have 
    \[2^{10} + 3^{15} = 14,349,931 = \underbrace{31\cdot 462\cdot 901}_{\text{not powerful}}\]
\end{example}
What about another example, like $3^{15} + 5$? The $abc$ conjecture also predicts that this number is not so powerful...\footnote{After lecture Jiahua: It's a prime!?}