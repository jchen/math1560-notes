%!TEX root = ../notes.tex
\section{March 24, 2022}
\subsection{Algebraic Integers \emph{continued}}
\recall our definition for algebraic integers\dots
\begin{definition*}[Algebraic Integer]
    A complex number that satisfies $f(x) = 0$ for a non-constant \emph{monic} polynomial $f(x)\in\ZZ[x]$ is an \ul{algebraic integer}.

    An algebraic number is an algebraic number whose minimal polynomial over $\QQ$ has integer coefficients.

    We denote this set by $\ZZbar$.
\end{definition*}

Clearly, we have that $\ZZbar \subseteq \QQbar$.
\begin{claim*}
    We want to show that $\ZZbar$ is, in fact, a subring of $\QQbar$.
\end{claim*}

\begin{lemma}[Setup Lemma, p.44 \cite{stewart2015algebraic}]
    $\theta\in\CC$ is an algebraic integer iff the additive group generated by all powers $1, \theta, \theta^2, \dots$, $\ZZ[\theta, \theta^2, \dots]$, is finitely generated.
\end{lemma}
\begin{proof}
    \begin{description}
        \item[Forward Direction.] Suppose $\theta\in\ZZbar$. Then for some $n$,
            \[\theta^n  + a_{n-1}\theta^{n-1} + \cdots + a_0 = 0,\]
            where $a_i\in\ZZ, \forall\ 0\leq i\leq n-1$.
            \begin{claim*}
                Every power of $\theta$ lies in the additive group $\Gamma$ generated by $1, \theta, \theta^2, \dots, \theta^{n-1}$.
            \end{claim*}
            Suppose inductively that $m\geq n$, and that $1, \theta, \theta^2, \dots, \theta^m\in\Gamma$. We express
            \begin{align*}
                \theta^{n+1} = \theta^{m+1-n}\theta^n & = \theta^{m+1-n}(-a_{n-1}\theta^{n-1} - \cdots - a_0)     \\
                                                      & = -a_{n-1}\theta^m - \text{lower degree stuff} \in \Gamma
            \end{align*}
        \item[Backward Direction.] Suppose every power of $\theta$ lies in a finitely generated additive group $G$. Then the subgroup $\Gamma$ of $G$ generated by $\left\{ 1, \theta, \theta^2, \dots \right\}$ must also be finitely generated.

            Let $v_1, \dots, v_n$ be generators of $\Gamma$. (\textsc{wlog} can assume not all zero). Each $v_i \in\ZZ[\theta]$ (polynomial in $\theta$ with integer coefficients), so $\theta v_i\in\ZZ[\theta]\ \forall i$. Hence there exists integers $b_{ij}$ such that
            \[\theta v_i = \sum_{j=1}^n b_{ij}v_j\quad \forall i.\]
            This gives us a system of linear equations
            \begin{align*}
                (b_{11}-\theta)v_1 + b_{12}v_2 + \cdots + b_{1n}v_n & = 0    \\
                b_{21}v_1 + (b_{22}-\theta)v_2 + \cdots + b_{2n}v_n & = 0    \\
                                                                    & \vdots \\
                b_{n1}v_1 + b_{n2}v_2 + \cdots + (b_{nn}-\theta)v_n & = 0
            \end{align*}
            So now we have $A\vec{v} = \vec{0}$ so $\det{A} = 0$.

            The $v_1, \dots, v_n\in \CC$ give a nontrivial solution to the obvious associated system of linear equations, so the determinant
            \[\left|\begin{array}{cccc}
                    b_{11} - \theta & b_{12}          & \cdots & b_{1n}          \\
                    b_{21}          & b_{22} - \theta & \cdots & b_{2n}          \\
                    \vdots          & \vdots          & \ddots & \vdots          \\
                    b_{n1}          & b_{n2}          & \cdots & b_{nn} - \theta
                \end{array}\right|\]
            is zero. So the determinant, expanding as minors as a polynomial, is a monic\footnote{The highest degree of $\theta$ comes from the diagonal which monic up to sign. We also have that this is the characteristic of the $b_{ij}$ matrix which is monic.} polynomial (in $\theta$) with integral entries $b_{ij}$ of which $\theta$ satisfies. So $\theta$ is an algebraic integer.
    \end{description}
    Both directions of which are as desired.
\end{proof}

Note we prove something stronger and more intuitive:
\begin{lemma*}
    $\theta\in\CC$ is an algebraic integer iff the additive subgroup generated by $1, \theta, \theta^2, \dots$ is in fact generated by $1, \theta, \theta^2, \dots, \theta^{n-1}$ for some $n$.
\end{lemma*}
\begin{theorem}
    $\ZZbar$ is a subring of $\QQbar$.
\end{theorem}
\begin{proof}
    Suppose that $\theta, \phi\in\ZZbar$. We want to show that $\theta + \phi, \theta\phi\in\ZZbar$.

    By the lemma, all powers of $\theta$ lie in a finitely generated subgroup $\Gamma_\theta$ of $\CC$ and similarly, all powers of $\phi$ lie in a finitely generated subgroup $\Gamma_\phi$ of $\CC$.

    \otoh, all powers of $\theta + \phi$ and $\theta\phi$ are integer linear combinations of the elements
    \[\theta^k \phi^l\in \Gamma_\theta\Gamma_\phi\]
    where $\Gamma_\theta\Gamma_\phi := $ the additive group generated by $v_iw_j$ where $1\leq i\leq n$ and $1\leq j\leq m$ with
    \begin{align*}
        \Gamma_\theta & = \langle v_1, \dots, v_n\rangle \\
        \Gamma_\phi   & = \langle w_1, \dots, w_m\rangle
    \end{align*}
    We note that $\Gamma_\theta\Gamma_\phi$ is finitely generated, and since each power of $\theta+\phi$ and $\theta\phi$ lie in this finitely generated subgroup\footnote{The subgroup that they generate had better be finitely generated.}, by our lemma $\theta + \phi$ and $\theta\phi$ are both algebraic integers.
\end{proof}

\begin{theorem}[p.44 or p.45 \cite{stewart2015algebraic}]
    Let $\theta\in\CC$ satisfy a monic polynomial equation with coefficients in $\ZZbar$ (not just in $\ZZ$). Then $\theta$ is an algebraic integer.
\end{theorem}
\begin{proof}
    \emph{One imitates the proof of the forward direction in our previous setup lemma, applying a bit of module theory.}
\end{proof}

\subsection{Ring of Integers of a Number Field}
\begin{definition}[Ring of Integers of Number Field $K$]
    If $K$ is a number field, then
    \[\riO_K := K\cap \ZZbar\]
    is called the \ul{ring of integers of $K$}.\footnote{In textbooks, it's \emph{fraktor} $\mathfrak{O}$. In papers, usually mathcal $\mathcal{O}$. In handwriting, usually fancy loopy $O$.}
\end{definition}
$\riO_K$ is a ring because $K$ and $\ZZbar$ are subrings of $\CC$. The relationship between $K$ and $\riO_K$ is the same as that of $\QQ$ and $\ZZ$.

\begin{lemma}
    If $\alpha\in K$, then $c\alpha\in \riO_K$ for some $c\in\ZZ$.
\end{lemma}
\begin{proof}
    Let $\alpha\in K$ and $f(x) = \minpoly_\QQ(\alpha)$, with $\deg f = n$. Let $0\neq c\in \ZZ$ and let $g_c := c^n\cdot f\left( \frac{x}{c} \right)$.

    Observe:
    \begin{enumerate}[1)]
        \item The roots of $g_c$ are the $c\alpha_i$ where $\alpha_i$ are the roots of $f$.
        \item $g_c$ is monic.
        \item If we choose $c$ to be the lcm of the denominators of the coefficients of $f$ implies that $g_c$ has integer coefficients.
    \end{enumerate}
    So $c\alpha$ is an element of $\riO_K$ since it is also an algebraic integer.
\end{proof}

\begin{corollary}
    If $K$ is a number field, then $K = \QQ(\theta)$, for some algebraic integer $\theta\in\ZZbar$.
\end{corollary}

\textbf{Warning!} (pp.46-47 \cite{stewart2015algebraic}) Though it is often the case that if $K = \QQ(\theta)$ with $\theta\in\ZZbar$, then $\riO_K = \ZZ[\theta]$, this need not be true.
\begin{example}
    Let $K = \QQ(\sqrt{5})$. However,
    \[\ZZ[\sqrt{5}]\subsetneq \riO_K\]
    In fact, \[\ZZ\left[ \frac{1+\sqrt{5}}{2} \right] = \riO_K.\]
\end{example}
Furthermore, is it always the case that $\riO_K$ is generated by a single element? \emph{No!} $\riO_K$ need \ul{not} be of the form $\ZZ[\theta]$ for some $\theta\in\ZZbar$.
\begin{example}
    The counterexample of this is
    \[K = \QQ(\theta)\]
    when $\theta$ is a root of $x^3 - x^2 - 2x - 8$.
\end{example}
Number fields where such a $\theta$ does exist are called \ul{monogenic}.