%!TEX root = ../notes.tex
\section{March 3, 2022}
\subsection{Quadratic Residues \emph{continued}}

\recall \cref{defn:legendre-symbol} and \cref{prop:5.1.2} from last class (right above).

We now prove the earlier proposition:
\begin{proof}[Proof of \cref{prop:5.1.2}]
    ~\begin{enumerate}
        \item[(c)] is clear.
        \item[(a)]
            By Fermat's Little Theorem, if $p\nmid a$, we have $a^{p-1}\equiv 1\mod p$, so
            \[(a^{(p-1)/2}+1)(a^{(p-1)/2}-1)\equiv 0\mod p\]
            Since mod $p$ we have an integral domain, then we have that $a^{(p-1)/2}\equiv \pm 1\mod p$. We know that $a^{(p-1)/2}\equiv 1\mod p$ if and only if $a$ is a quadratic residue mod $p$,
        \item[(b)]
            This applies (a).
            \[\lege{ab}{p} = (ab)^{(p-1)/2} = a^{(p-1)/2}\cdot b^{(p-1)/2} = \lege{a}{p}\lege{b}{p}\]
    \end{enumerate}
\end{proof}
\begin{corollary}
    We have some corollaries:
    \begin{enumerate}[1)]
        \item There are exactly $\frac{p-1}{2}$ quadratic residues and $\frac{p-1}{2}$ quadratic non-residues mod $p$.
        \item The product of two residues is a residue, the product of a residue and a non-residue is a non-residue, and the product of a non-residue and a non-residue is a residue.
        \item If $g$ is a primitive root modulo $p$, then
              \[\lege{g^i}{p} = (-1)^i.\]
        \item We have
              \[\lege{-1}{p} = (-1)^{(p-1)/2}\]
              which has a fancy name. This is the \emph{``First Supplemental Law of Quadratic Reciprocity''}.
    \end{enumerate}
\end{corollary}

\subsection{Gauss's Lemma}
We now discuss a characterization of the Legendre symbol due to Gauss.
\begin{definition}
    For $p\in\ZZ_+$ an odd prime,
    \[S = \left\{-\frac{p-1}{2}, -\frac{p-3}{2}, \dots, -1, 1, 2, \dots, \frac{p-1}{2}\right\}\]
    is called the \ul{set of least residues mod $p$}.
\end{definition}

\begin{definition}
    Let $a\in\ZZ$ such that $p\nmid a$. Define $\mu$ to be the number of negative least residues of the integers
    \[a, 2a, 3a, \dots, \left(\frac{p-1}{2}\right)a\]
\end{definition}

\begin{example}
    If $p=7$ and $a=4$, then $\frac{p-1}{2} = 3$, and
    $1\cdot 4, 2\cdot 4, 3\cdot 4$ are congruent to $-3, 1, -2$ mod $7$. Thus $\mu = 2$.
\end{example}
\begin{lemma}[Gauss's Lemma]
    Let $p\in\ZZ_+$ be an odd prime and let $a\in\ZZ$ be such that $p\nmid a$. Then
    \[\lege{a}{p} = (-1)^\mu.\]
\end{lemma}
\begin{proof}
    It is convenient for us to partition the list $S$ as
    \begin{align*}
        P & = \{1, 2, \cdots, \frac{p-1}{2}\}    \\
        N & = \{-1, -2, \cdots, -\frac{p-1}{2}\}
    \end{align*}
    so that $\mu = |aP\cap N|$\footnote{There's an abuse of notation here, but we conflate integers with their equivalence classes in $S$.}.

    A key observation is that if $x, y\in P$ with $x\neq y$, then
    \[ax\not\equiv \pm ay\mod p\]
    for otherwise,
    \[a \equiv \pm y\mod p\]
    which is impossible since $x$ and $y$ are distinct elements of $P$ (positive residue classes, so not in $N$).

    Thus $aP = \{\varepsilon_i i\mid 1\leq u\leq \frac{p-1}{2}\}$ for some $\varepsilon_i = \pm 1$.

    Now we mimic the elementary proof of Euler's Theorem:
    \begin{align*}
        a^{(p-1)/2}\cdot \left(\frac{p-1}{2}\right)! & \equiv \left(\prod_{i=1}^{(p-1)/2} \varepsilon_i\right)\cdot \left(\frac{p-1}{2}\right)! \\
        a^{(p-1)/2}                                  & \equiv \left(\prod_{i=1}^{(p-1)/2} \varepsilon_i\right)                                  \\
                                                     & \equiv (-1)^\mu
    \end{align*}
    since $\mu$ is equal to the order of the set that is contributing every $-1$ as $\varepsilon_i$.

    Applying \cref{prop:5.1.2} (Euler's criterion), this concludes the proof.
\end{proof}
We'll use this lemma to prove Quadratic Reciprocity later.

We now use it to prove the \emph{Second Supplemental Law} of Quadratic Reciprocity.
\begin{proposition}[5.1.3, the \emph{Second Supplemental Law of Quadratic Reciprocity}]
    For $p\in\ZZ_+$ be an odd prime.
    \[\lege{2}{p} = (-1)^{(p^2 - 1)/8}\]
    We note $\dfrac{p^2-1}{8} = \dfrac{(p-1)(p+1)}{2\cdot 4}$. From this, it follows that we're really saying
    \[\lege{2}{p} = \begin{cases}
            1  & \text{when $p = \pm 1\mod 8$} \\
            -1 & \text{when $p = \pm 3\mod 8$}
        \end{cases}\]
\end{proposition}
\begin{proof}
    We apply Gauss's Lemma with $a = 2$.
    \[2P = \{2, 4, 6, \dots, p-1\}\]
    First suppose that $p\equiv 1\mod 4$. Then $\frac{p-1}{2}$ is even, so
    \[2P = \bigg\{\underbrace{2, 4, 6, \dots, \frac{p-1}{2}}_{\in P}, \underbrace{\frac{p+3}{2}, \dots, p-1}_{\in N}\bigg\}\]
    with the first $\frac{p-1}{4}$ elements in $P$ and the last $\frac{p-1}{4}$ elements in $N$. So $\mu = |2P\cap N| = \frac{p-1}{4}$, so Gauss's Lemma gives
    \[\lege{2}{p}
        = (-1)^{\frac{p-1}{4}}
        = \left((-1)^{\frac{p-1}{4}}\right)^{\frac{p+1}{2}}
        = (-1)^{\frac{p^2-1}{8}}\]
    since $\frac{p+1}{2}$ is odd.

    Now suppose $p\equiv 3\mod 4$. Then
    \[2P = \bigg\{\underbrace{2, 4, 6, \dots, \frac{p-3}{2}}_{\in P}, \underbrace{\frac{p+1}{2}, \dots, p-1}_{\in N}\bigg\}\]
    The first $\frac{p-3}{4}$ elements are in $P$, and the last $\frac{p+1}{4}$ elements are in $N$. Then $\mu = \frac{p+1}{4}$, so
    \[\lege{2}{p}
        = (-1)^{\frac{p+1}{4}}
        = \left((-1)^{\frac{p+1}{4}}\right)^{\frac{p-1}{2}}
        = (-1)^{\frac{p^2-1}{8}}\]
\end{proof}

\subsection{Quadratic Reciprocity}
\begin{theorem}[Law of Quadratic Reciprocity]
    Let $p, q\in\ZZ_+$ be distinct odd positive primes. Then
    \begin{equation*}
        \lege{p}{q}\lege{q}{p} = (-1)^{\frac{p-1}{2}\frac{q-1}{2}}
    \end{equation*}
    In other words,
    \[\lege{p}{q} = \lege{q}{p}\]
    if and only if at least one of $p, q$ is congruent to $1$ mod $4$.
\end{theorem}
\begin{proof}
    \emph{coming soon!}
\end{proof}

Here's the motivation behind this: we used Euler's Criterion to easily calculate the quadratic character of some $a$ mod $p$. With Quadratic Reciprocity, we can solve the question ``If we fix odd prime $q$, for which $p$ is $q$ a quadratic residue?''

\begin{example}
    Which odd primes $p\in\ZZ_+$ have $3$ as a quadratic residue?

    Suppose $p\equiv 1\mod 4$. Then
    \[\lege{3}{p} = \lege{p}{3} = \begin{cases}
            1  & \text{if $p\equiv 1\mod 3$}  \\
            -1 & \text{if $p\equiv -1\mod 3$}
        \end{cases}\]
    Then $p\equiv 1\mod 4$ and $p\equiv 1\mod 3$ so $p\equiv 1\mod 12$ gives us $\lege{3}{p} = 1$ and $p\equiv 5\mod 12$ gives $\lege{3}{p} = -1$. ***(format)

    Now suppose $p\equiv 3\mod 4$. Then
    \[\lege{3}{p} = -\lege{p}{3} = \begin{cases}
            1  & \text{if $p\equiv -1\mod 3$} \\
            -1 & \text{if $p\equiv 1\mod 3$}
        \end{cases}\]
    Thus,
    \begin{align*}
        p & \equiv 11\mod 12\text{ gives }\lege{3}{p} = 1\text{, and } \\
        p & \equiv 7\mod 12\text{ gives }\lege{3}{p} = -1
    \end{align*}

    So we conclude that $\lege{3}{p} = 1$ iff $p\equiv \pm 1\mod 12$.
\end{example}